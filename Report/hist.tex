\section{Historical Cylindrical Models (1977-2000)}\label{hist_sec}

Mostly frequentist models essentially starting with Mardia and Sutton \cite{mardia_model_1978} and Johnson and Wehrly \cite{johnson_angular-linear_1978} in 1978. These models have been extended and built upon mainly in the frequentist setting up to the present. These two lineages form the core of cylindrical data analysis, and will be the basis of the following two subsections. Most papers up to 2000 seem to stem from one of the two foundational papers. This section concludes with the extension to a second order model by Anderson-Cook \cite{anderson-cook_second_2000} in 2000. After this paper, the cylindrical model literature becomes more Bayesian in focus and a greater emphasis is placed on the joint relationship between $\theta$ and $Z$. In addition, this time period was mainly focused on models with the von Mises distribution for the circular marginal distribution. While several other models were proposed by Gould (1969) \cite{gould_regression_1969} and Laycock (1975) \cite{laycock_optimal_1975}, these were never widely adopted due to serious issues with their specification preventing them from being usable in practice.

\subsection{Johnson and Wehrly Model Lineage}\label{hist_JW}

In section \ref{hist_JW}, we first consider the work from Johnson and Wehrly in 1977 \cite{johnson_measures_1977} and 1978 \cite{johnson_angular-linear_1978} which define population models for dependence between linear and circular random variables and a distribution on the cylinder based on maximum entropy respectively. Fisher and Lee then extended the Johnson and Wehrly model in their papers proposing a model for circular-linear regression and circular time series \cite{fisher_regression_1992} \cite{fisher_time_1994}. Further work in lineage picks up with Fern\'{a}ndez-Dur\'{a}n in 2007 \cite{fernandez-duran_models_2007} in section \ref{mod_JW}. 

In 1977, Johnson and Wehrly proposed population models for dependence between linear and circular random variables \cite{johnson_measures_1977}. They also proposed a few sample measurements of the correlation. The canonical correlation method from multivariate analysis is used to create the measures for both circular-circular and circular-linear correlations, which unifies several ad hoc methods. The canonical correlation method considers the circular r.v. $\theta$ as cartesian coordinates jointly with the linear random variable $Z$ to have the multivariate vector $(Y_1, Y_2, Y_3)' = (\cos\theta, \sin\theta, Z)$ with observations $(y_1, y_2, y_3)'$ and uses the canonical correlation framework from there to produce 
\begin{align*}
    r_{CL}^2 &= \frac{s_{11} s_{23}^2 + s_{22} s_{13}^2 - 2s_{12}s_{23}s_{31}}{s_{33}(s_{11}s_{22} - s_{12}^2)} \\
    Var(r_{CL}) &= r_{CL}^{-2} \sum_i \sum_j \frac{\partial^2 r_{CL}^2}{\partial s_{ij}\partial s_{km}} cov(s_{ij}, s_{km}) \\
    \rho(X, \cos(\theta - \alpha) &= \frac{-\sqrt{2} \sin(\mu_1 - \alpha)\sigma_{12}\exp(-\sigma_1^2 / 2)}{\sigma_2 [(1 - \exp(-\sigma_1^2)\cos(2(\mu_1 - \alpha))(1 - \exp(-\sigma_1^2)]^{1/2}}
\end{align*}
\[
\max_\alpha \rho_{CL}(X, \cos(\theta - \alpha) =\begin{cases}
    \rho(cos(\theta - \mu_1 - \frac{\pi}{2}, Z), & \sigma_{12} > 0 \\
    \rho(cos(\theta - \mu_1 + \frac{\pi}{2}, Z), & \sigma_{12} < 0
    \end{cases}
\] where $s_{ij}$ is the sample covariance between $Y_i$ and $Y_j$, $(X_1, Z) \sim N_2\left(\begin{pmatrix}\mu_1 \\ \mu_2 \end{pmatrix}, \begin{pmatrix}\sigma_1^2 & \sigma_{12} \\ \sigma_{12} & \sigma_2^2 \end{pmatrix}\right)$ and $\theta = X_1 (\mod 2\pi)$. Inference on the correlations is performed via large sample approximations; $r \dot{\sim} N(\rho_{CL}, Var(r_{CL}))$. 

Building on their work, Johnson and Wehrly then proposed their foundational cylindrical model in 1978 \cite{johnson_angular-linear_1978}. Three maximum entropy densities on the cylinder are introduced with increasing levels of complexity. The third, most complex model allows for a von Mises circular marginal and exponential linear marginal when $\theta$ and $Z$ are independent ($\kappa = 0$). The density is as follows \begin{equation*}
\begin{split}
f(\theta, z) = &(2\pi)^{-1}(\lambda^2 - \kappa^2)^{1/2} \left[I_0(\nu) + 2\sum_{p=1}^\infty \rho^p I_p(\nu) \cos(p(\mu_1 - \mu_2))\right]^{-1} \\
&\times \exp\left[-\lambda z + \kappa z \cos(\theta - \mu_1) + \nu \cos (\theta - \mu_2)\right]
\end{split}
\end{equation*} where $\rho = \kappa(\lambda + \sqrt{\lambda^2 - \kappa^2})^{-1}$ and $\theta \in (0, 2\pi]$, $Z > 0$, the circular parameters $\mu_1, \mu_2 \in (0,2\pi]$, the dependence parameter $\kappa \in (0,\lambda)$, and $\lambda > 0$ is the exponential scale parameter in the independent case. Generally, the marginals do not have a closed form. 

Of particular interest are the conditional distributions $\theta | Z = z \sim M(\mu^*, \kappa^*)$ and $Z | \theta \sim Exp((\lambda - \kappa \cos(\theta - \mu_1)^{-1})$ with $E(Z|\theta) = (\lambda - \kappa \cos(\theta - \mu_1)^{-1}$. $\mu^*$ and $\kappa^*$ are defined by the relationship \begin{align*}
    \kappa^* \cos \mu^* &= \kappa x \cos\mu_1 + \nu \cos \mu_2 \\
    \kappa^* \sin \mu^* &= \kappa x \sin\mu_1 + \nu \sin \mu_2 
\end{align*} These conditional distributions then facilitate regression models as were later developed by Fisher and Lee \cite{fisher_regression_1992}. 

Two regression models for circular $\theta$ on a linear covariate $Z$ were developed in \cite{johnson_angular-linear_1978}. A previous model of this type was proposed by Gould (1969) \cite{gould_regression_1969} but had serious flaws (infinitely many equivalent maxima, no link function, potentially periodic likelihood). The first of the Johnson and Wehrly models for regression has centered angles; $\theta | Z = z \sim M(\mu + 2\pi F(z), \kappa)$. The second has the covariates be scale parameters of the conditional distribution of the angle given the covariates; $\theta | Z = z \sim M(\mu, z\kappa)$. Both models produce von Mises conditional distributions, a feature that was highly desired during this period of directional statistics. The parameters of both models can then be computed via maximum likelihood and inference can be performed asymptotically through the normal approximation of the MLEs. To conclude their paper, a trigonometric regression model is discussed for regression $Z$ on $\theta$.

The circular-linear regression model from Johnson and Wehrly \cite{johnson_angular-linear_1978} was further developed and extended by Fisher and Lee in 1992 \cite{fisher_regression_1992}. Fisher and Lee bring the models into a GLIM-like framework via link functions such as $g(x) = tan^{-1}(x)$. This theoretically allows for a set of covariates rather than a single covariate as in the Johnson and Wehrly models \cite{johnson_angular-linear_1978} and Mardia and Sutton model \cite{mardia_model_1978}. Three major drawbacks of this approach are the lack of consideration of the joint distribution, the difficulties of actually implementing the Fisher and Lee model with multiple covariates, and the choice to ignore the circular nature of $\mu$ in the asymptotic inference via normal approximation. 

Three extensions were introduced by Fisher and Lee \cite{fisher_regression_1992}. The first extension is of the model $\theta | Z = z \sim M(\mu + 2\pi F(z), \kappa)$ to $\theta | \mathbf{Z} = \mathbf{z} \sim M(\mu + 2\pi g(\mathbf{z}'\bm{\beta}), \kappa)$ which allows for multiple linear covariates (or, by borrowing from Mardia and Sutton's model \cite{mardia_model_1978}, circular covariates via $(Z_{j1}, Z_{j2}) = (\cos\eta, \sin\eta)$) and for more flexibility in how the covariates are linked to the mean direction of $\theta$. Since this model is so similar the GLIM, standard IRLS can be used to find MLEs for the model and then inference can be done with the asymptotically normal distributions of the estimates. 

For the second model from Johnson and Wehrly \cite{johnson_angular-linear_1978}, Fisher and Lee propose an extension with a link function $h(.)$ for the concentration parameter $\kappa$; $\theta | \mathbf{Z} = \mathbf{z} \sim M(\mu, \kappa_i)$ where $\kappa_i = h(\mathbf{z}'\bm{\gamma})$ \cite{fisher_regression_1992}. Their proposal was to use $h(x) = \exp(x)$. The final extension offered by Fisher and Lee is to use a model combining the other two's properties so that both the mean direction and concentration of $\theta$ varies according to $\mathbf{Z}$ \cite{fisher_regression_1992}. Both of these models can also be estimated using IRLS and inference again performed with the asymptotic normality of the MLEs. The major step forward offered by Fisher and Lee was the consideration of $\theta$ varying in both direction and concentration according to $\mathbf{Z}$. 

% As of 1992, two kinds of circular times series had been considered in the literature: (i) Markov process models, (ii) wrapped linear time series \cite{fisher_time_1994}. This paper focuses on the wrapped AR process originally proposed by \textcolor{red}{Breckling 1989}, a projected process, and a link function process extended from the Fisher and Lee 1992 paper \cite{fisher_regression_1992}. This model does not consider any covariate or magnitude associated with the directions. The circular time series can be considered as a cylindrical model that 

% Projection method allows for analysis of circular TS with uniform marginal distributions. Fitting the projected model is framed a missing data problem (find $x,y$ instead of the angle?), letting us use EM to fit the model. The variances of the latent $(X,Y)$ (when equal) don't matter to the projected  process. 

% Wrapped method takes a univariate TS $\mod 2\pi$. If we only collect the angles, then recovering the $X_t$ for modelling is again a missing data problem. The wrapped Gaussian AR(p) (WAR(p)) process seems to be easy to work with for low order p processes. 

% Link function processes (LARMA(p,q)) is a circular stationary process when the linked linear process $g^{-1}(\theta_t -  \mu)$ is ARMA(p,q) (reverse the mapping of the means). Alternatively it can be defined conditionally as a von Mises process (inverse AR(p)).  

% What’s convenient for the methods that transform back to $X_t$ is the ability to use standard TS methods. 

% [Maybe this is the approach to take with the MI problem as well.]   

\subsection{Mardia and Sutton Model Lineage}\label{hist_MS}

The Mardia and Sutton lineage of cylindrical models begins with their 1978 paper proposing a joint density on the cylinder along with a proposed first order regression of $Z$ on $\theta$ and a test of independence \cite{mardia_model_1978}. The proposed joint density is a product of a von Mises and Normal random variables \[f(z,\theta) = [2\pi I_0(\kappa)]^{-1} \exp[\kappa \cos(\theta - \mu_0)][2\pi \sigma_c^2]^{-1/2} \exp\left[-\frac{(z - \mu_c)^2}{2\sigma_c^2}\right]\] with related location parameters $\mu_0 \in (0, 2\pi]$ and $\mu \in \mathbb{R}$, scale parameters $\kappa > 0$ and $\sigma^2 \in \mathbb{R^+}$ linked by $\rho_1, \rho_2 \in [0,1]$ which is the parameter of dependence between $Z$ and $\theta$. \begin{align*}
    \mu_c &= \mu + \sigma \kappa^{1/2} [\rho_1 (\cos\theta - \cos\mu_0) + \rho_2 (\sin\theta - \sin\mu_0)] \\
    \sigma_c^2 &= \sigma^2(1 - \rho^2) & \rho &= \sqrt{\rho_1^2 + \rho_2^2}
\end{align*} $Z$ given $\theta$ is conditionally normal with $E(Z | \theta) = b_0 + b_1 \cos\theta + b_2 \sin\theta$ which we can call the first order model. Marginally $\theta$ is von Mises with mean direction $\mu_0$ and concentration parameter $\kappa$. MLEs and a LRT of independence are developed. The application at the end of the paper uses a brief data set of wind directions and surface temperatures which is commonly used a reference data set in the study of cylindrical models and referred to as the Kew data set \cite{anderson-cook_second_2000}. The main two uses of the methods developed are to investigate the dependence between a linear r.v. and a directional r.v. and create a regression model that respects the properties of the direction with direction as a (random) covariate and the linear r.v. as a response. Note that the direction of the regressions is opposite that of the Johnson and Wehrly models \cite{johnson_angular-linear_1978} \cite{fisher_regression_1992}.

Anderson-Cook published her first extension of the Mardia cylindrical model in 1997 \cite{anderson-cook_extension_1997}. She introduces C-association in order to make the conditional density $Z|\theta$ more flexible. She also introduces an LRT to see if the new model is a better fit over the Marida and Sutton model \cite{mardia_model_1978}. The Mardia and Sutton model is C-Linear meaning the relationship can be described by a single sine or cosine curve. \[\mu_c = a_0 + a_1 \cos(\theta - \theta_0) = B_0 + B_1 \cos \theta + B_2 \sin\theta\] C-association relaxes the symmetry required of the sine and cosine curves. C-association was introduced by Fisher in 1993 \cite{fisher_statistical_1995} and requires that there is a single minimum and maximum (not necessarily $\pi$ radians apart) over $(0,2\pi]$ with the same functional value at the endpoints. The new joint density is \[f(z,\theta) = [2\pi I_0(\kappa)]^{-1}\exp(\kappa\cos(\theta - \mu_0)) [2\pi \sigma^2_c]^{-1/2}\exp\left[-\frac{(z - \mu_c)^2}{2\sigma_c^2}\right]\] where \[\mu_c =\begin{cases}
b_0 + b_1 \cos(\pi\frac{\theta - \alpha}{\beta - \alpha}), & \alpha \leq \theta \leq \beta \\
b_0 - b_1 \cos(\pi\frac{\theta - \alpha}{\alpha - \beta}), & \beta \leq \theta \leq \alpha 
\end{cases}\] is defined as above and $\sigma_c^2$ is the variance of $Z|\theta$.

The directional marginal remains von Mises and the conditional $Z|\theta$ remains normal \cite{anderson-cook_extension_1997}. There’s an additional parameter added into the definition of the mean and variance of the normal compared to the Mardia and Sutton model \cite{mardia_model_1978}. The model is most applicable when the circular observations cover the entire support of the unit circle. Closed forms do not exist for the parameter estimates, so numerical MLE methods must be used such as Newton-Raphson \cite{anderson-cook_extension_1997}. The numerical methods are very dependent on the initial value used to start the algorithm since the likelihood is irregular, so starting close to the global maximum is essential. Similar issues exist for the likelihood of the Fisher and Lee regression model \cite{fisher_regression_1992}.

In 2000, Anderson-Cook offered a second order extension of the Marida and Sutton first order model \cite{anderson-cook_second_2000} \cite{mardia_model_1978}. The new model sets the conditional expectation \[E(Z |\theta) = b_0 + b_1 \cos\theta + b_2\sin\theta + b_{11}\cos2\theta + b_{22}\sin2\theta\] The second order model allows for a bi-modal relationship between the components. The covariates can be expressed in cartesian coordinates or polar coordinates. Cartesian coordinates allow for simpler computation while the polar coordinates are conceptually easier to work with. Note that the first order model is equivalent to $y = b0 + b1 x$ while second order model is equivalent to $y = b0 + b1 x + b2 x^2$ since the new conditional expectation can be simplified to the quadratic model via the double angle identities. However the second order model is intractable since adding quadratic and interaction terms of cosine and sine would make the design matrix less than full rank. Model fitting is simply LS \cite{anderson-cook_second_2000}.

Inference to choose between models can be performed using the nested F-test since the first order model is nested within the second order model. Inferences on individual regression coefficients can be performed as usual as well \cite{anderson-cook_second_2000}. Bayesian extensions are not mentioned in the paper, but since this is essentially a standard linear model, the standard implementation of a normal linear regression would apply as well allowing for Bayesian inference as well.

Following the Anderson-Cook (2000) paper, the literature focuses more on the joint distribution of $(Z, \theta)$ with a broader scope of methods through the introduction of Bayesian \cite{sadeghianpourhamami_bayesian_2019}\cite{abe_tractable_2017}\cite{mulder_bayesian_2017}\cite{mastrantonio_joint_2018}, non-parametric \cite{carnicero_non-parametric_2013}\cite{garcia-portugues_exploring_2013}, and semi-parametric analysis \cite{fernandez-duran_models_2007}. While the two core Mardia and Sutton lineages continue to be developed, a new distribution is proposed by Abe and Ley \cite{abe_tractable_2017}.
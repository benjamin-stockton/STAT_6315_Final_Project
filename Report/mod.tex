\section{Modern Cylindrical Models (2000-Present)}\label{mod_sec}

The models considered in this section have been developed in the Bayesian, the frequentist, and the non-parametric settings with a focus on flexibility and generalizability. The von Mises is less central while the Projected Normal becomes more popular. Mixture distributions and skewed variants of the von Mises appear more often. Much of the work done over the past 20 years has been in the Johnson and Wehrly lineage \cite{johnson_angular-linear_1978} or by extension from the Fisher and Lee GLM models \cite{fisher_regression_1992}\cite{fernandez-duran_models_2007}\cite{carnicero_non-parametric_2013} \cite{garcia-portugues_exploring_2013}\cite{nunez-antonio_bayesian_2011}\cite{mulder_bayesian_2017}. These models were developed in all three frameworks. The developments in the Mardia and Sutton lineage \cite{mardia_model_1978} largely focused on generalizing the model to make it more flexible beyond what the Mardia and Sutton \cite{mardia_model_1978} and Anderson-Cook models \cite{anderson-cook_extension_1997}\cite{anderson-cook_second_2000} allowed and on the joint distribution since the conditional $Z|\theta$ regression is largely fully covered by the GLM literature. Aside from the two central lineages, the Abe-Ley distribution was introduced in the past 5 years along with fully Bayesian extensions \cite{abe_tractable_2017}\cite{sadeghianpourhamami_bayesian_2019}, new cylindrical clustering methods were offered \cite{}, and further work on the cylindrical time series was published \cite{lagona_hidden_2015}.

% Models up to present
\subsection{Johnson and Wehrly Lineage}\label{mod_JW}
% \subsection*{Models for Circular-Linear and Circular-Circular Data Constructed from Circular Distributions Based on Nonnegative Trigonometric Sums }- Fernández-Durán (2007 – Journal of the International Biometric Society) 

Before discussing the papers in this section, I'll give a brief overview of copulas as they are used in \cite{fernandez-duran_models_2007}\cite{garcia-portugues_exploring_2013}\cite{carnicero_non-parametric_2013}. A bivariate copula $C(.,.)$ is a distribution on $[0,1]^2$ s.t. both marginals are $U(0,1)$. Any joint distribution of $X,Y$ can be expressed as $F_{X,Y}(x,y) = C(F_X(x), F_Y(y))$. If the marginals are continuous the Copula is unique. The empirical copula is defined as  \[\hat{C}_n = \frac{1}{n} \sum_{I=1}^n I(u_i \leq u, v_i \leq v)\] where $(u_i, v_i) = (\hat{F}_X(x_i), \hat{F}_Y(y_i))$ \cite{carnicero_non-parametric_2013}. 

The first paper introducing semi-parametric methods builds on Johnson and Wehrly (1978) (and their 1980 paper on toroidal models) comes from Fern\'{a}ndez-Dur\'{a}n \cite{fernandez-duran_models_2007}. Their main goal was to propose a way to generate models from flexible marginals to allow for multimodality and skewness. The main tool they used was non-negative trigonometric sums. \[T(\theta) = a_0 + \sum_{k=1}^n \left[a_k \cos(k\theta) + b_k \sin(k\theta)\right]\] where for complex numbers $c_k$, $a_0 = \sum_{k=0}^n|c_k|^2$ and $a_k - i b_k = 2\sum_{v=0}^{n-k}c_{v+k} \bar{c}_v$. Fern\'{a}ndez-Dur\'{a}n defines a new circular distribution based on a non-negative trigonometric sum \[f(\theta; \mathbf{c}, n) = (2\pi)^{-1} + \pi^{-1} \sum_{k=1}^n \left[a_k \cos(k\theta) + b_k \sin(k\theta)\right]\] where $\sum_{k=0}^n|c_k|^2 = (2\pi)^{-1}$ and $a_k - i b_k = 2\sum_{v=0}^{n-k}c_{v+k} \bar{c}_v$ for complex numbers $c_k = c_{rk} + i c_{ck}$ so that there are $2n+2$ complex parameters (with constraints $c_{r0} > 0$. When $n=0$, you get the uniform circular distribution (allowing for independence in the cylindrical model), and $n=1$ gives the cardioid distribution. 

To define the cylindrical distribution, Fern\'{a}ndez-Dur\'{a}n makes use of copulas which Johnson and Wehrly implicitly use to define their maximum entropy model \cite{fernandez-duran_models_2007}\cite{johnson_angular-linear_1978}. The new joint density is \[f_{\theta, Z}(\theta, z) = c(F_\theta(\theta), F_Z(z))f_\theta(\theta)f_Z(z)\] where $c(u,v) = \frac{\partial^2 C(u,v)}{\partial u \partial v}$ is the copula density of the copula $C(.,.)$. The conditional distributions then allow for regression since $f_{Z|\theta}(z|\theta) = c(F_\theta(\theta), F_Z(z))f_Z(z)$ and $f_{\theta|Z}(\theta) = c(F_\theta(\theta), F_Z(z))f_\theta(\theta)$ while the marginals are simply $f_z$ and $f_\theta$ \cite{fernandez-duran_models_2007}. When using the circular-linear regression, estimation is performed using numerical maximum likelihood. Model selection is done with AIC and BIC while it is unclear how inference is completed. 

% Two applications are to a wind and pollution cylindrical model and to a toroidal model of dihedral angles between amino acids in a protein. The cylindrical model is referred to as a CL regression. The circular marginal density is the nonnegative trigonometric sum and the linear density is the Weibull. The explanatory variables (temp, RH\%, etc) go into the scale parameter of the Weibull density. Unclear how the actual inferences (ie what's significant) are reached.  

% Mixtures of von Mises in Mooney, Helms, and Jolliffe (2003) 

% \subsection*{Exploring Wind Direction and SO2 Concentration by Circular-Linear Density Estimation} - Garc\'{i}a-Portugu\'{e}s (2013 - Stochastic Environmental Research and Risk Assessment) 

% \textbf{Non-parametric Paper }

The next two papers both were published in 2013 and build on the non-parametric theory of copulas introduced into the literature by \cite{fernandez-duran_models_2007}. The first paper we'll discuss is from Garc\'{i}a-Portugu\'{e}s et al. This paper is focused on developing a model for wind direction and pollution. As in the Fern\'{a}ndez-Dur\'{a}n paper \cite{fernandez-duran_models_2007}, users can specify their marginal distributions either parameterically or empirically. The copula then combines them to generate a new joint density. Here the copula function is defined as in Johnson-Wehrly \cite{johnson_angular-linear_1978} \[c_{\theta, X}(u,v) = 2\pi g(2\pi(u\pm v))\] or as a Quadratic Section (QS copula) \[c_{\theta, X}^\alpha (u,v) = 1 + 2\pi \alpha \cos(2\pi u) (1 - 2v)\] $|\alpha| \leq (2\pi)^{-1}$ or a reflected copula \[c_{\theta, X} (u,v) = \frac{1}{4}(c^*(u,v) + c^*(1-u,v) + c^*(u,1-v) + c^*(1-u,1-v))\] $c^*$ is any copula, or as a Frank copula \[\tilde{c}_\alpha(u,v) = \frac{\alpha (1 - e^{-\alpha}) e^{-\alpha(u+v)}}{((1 - e^{-\alpha}) - (1 - e^{-\alpha u})(1 - e^{-\alpha v}))^2}\] for $\alpha \neq 0$ and where $(u, v) = (\Phi(\theta), F(x))$ for each copula. This specification requires symmetric dependence as noted by Carnicero \cite{carnicero_non-parametric_2013}. 

Estimation of the joint density and marginals is completed via kernel density estimation in this paper while maximum likelihood could also be used \cite{garcia-portugues_exploring_2013}. Garc\'{i}a-Portugu\'{e}s et al also discuss how data can simulated from the given joint density. Beyond the density estimation, inference is not discussed.

% \subsection*{Non-parametric Copulas for Circular-Linear and Circular-Circular Data: An Application to Wind Directions} - Carnicero et al (2013 - Stochastic Environmental Research and Risk Assessment) 

% \textbf{Non-parametric Paper }

Taking a step beyond Garc\'{i}a-Portugu\'{e}s et al, this paper from Carnicero et al uses Bernstein copulas to define a new non-parametric distribution on the cylinder and torus \cite{carnicero_non-parametric_2013}. They built on the Garc\'{i}a-Portugu\'{e}s et al non-parametric copula density by eliminating the requirement of symmetric dependence from the copula. This generalization is accomplished by using the Bernstein polynomial to define a Bernstein copula\cite{carnicero_non-parametric_2013}. The Bernstein polynomial $g(u,v)$ where $(u,v) \in [0,1]^2$ is defined as \[B_k(u,v) = \sum_{j_1 = 0}^k \sum_{j_2 = 0}^k g\left(\frac{j_1}{k}, \frac{j_2}{k}\right) {k \choose j_1} u^{j_1} (1 - u)^{k- j_1} {k \choose j_2} v^{j_2} (1 - v)^{k - j_2}\] which converges uniformly to $g(u,v)$ as $k \rightarrow \infty$. The Bernstein Copula is then defined as \[\hat{C}_B(u,v) = \sum_{j_1 = 0}^k \sum_{j_2 = 0}^k \hat{C}_n\left(\frac{j_1}{k}, \frac{j_2}{k}\right) {k \choose j_1} u^{j_1} (1 - u)^{k - j_1} {k \choose j_2} v^{j_2} (1 - v)^{k - j_2}\] where $\hat{C}_n(.,.)$ is defined as the empirical copula from the beginning of section \ref{mod_JW}. The Empirical Bernstein copula density is thus a mixture of beta densities \cite{carnicero_non-parametric_2013}.

The Bernstein copula is modified to accommodate the circular density by modifying the mixing proportion of the density. This allows for an asymmetric and more flexible distribution \cite{carnicero_non-parametric_2013}. The authors also propose a regression of $X | \theta$ based on this copula which is an analysis of primary interest. Toroidal models are developed simultaneously since we can simply replace the linear component with a similarly modified mixing proportion for a second circular density.  

To estimate the density, Carnicero et al use mixtures of the von Mises to estimate the circular marginal and mixtures of the exponential dist for the linear marginal. The estimated CDFs from the mixtures are then used to define the copula density. Again the inferences are limited to estimation of the distribution which in their example of wind and rainfall does allow for point estimation of mixture parameters\cite{carnicero_non-parametric_2013}. No uncertainty estimate is offered for the point estimates.

%% Mulder and Klugkist circglmbayes model

The final model I'll discuss in the Johnson and Wehrly lineage comes from Mulder and Klugkist \cite{mulder_bayesian_2017}. The authors build on Fisher and Lee's GLIM-like framework to propose what they call the circular GLM in the Bayesian framework, another circular-linear regression. Again, the von Mises is the desired distribution for $\theta$, with a very similar specification to the mean model from Fisher and Lee \cite{fisher_regression_1992}. $\theta_i | (\mathbf{z}_i, \mathbf{d}_i), \mu_0, \bm{\beta}, \bm{\gamma}, \kappa \sim M(\mu_i, \kappa)$ where $\mu_i = \mu_0 + \bm{\gamma}'\mathbf{d}_i + g(\bm{\beta}'\mathbf{z}_i)$ where $\mathbf{d}_i$ is a vector of dummy variables and given a link function $g(.)$ of the same kind as in Fisher and Lee \cite{mulder_bayesian_2017}\cite{fisher_regression_1992}. Note that this specification requires that the concentration parameter is constant throughout the range of $\theta$, equivalent to the assumption of homoscedasticity in the GLM. Pulling the dummy variables out of the link function lets the model better estimate relationships where we have "parallel" curves (ie the curves vary only by the value of $\mathbf{d}_i$) \cite{mulder_bayesian_2017}. 

To fully specify the model, Mulder and Klugkist use the latent variable $\psi_i = \theta_i - \bm{\gamma}'\mathbf{d}_i - g(\bm{\beta}'\mathbf{z}_i) \sim M(\mu_0, \kappa)$ \cite{mulder_bayesian_2017}. Then the priors are factored as $\pi(\mu_0, \kappa, \bm{\gamma}, \bm{\beta}) = \pi(\mu_0, \kappa| \bm{\gamma}, \bm{\beta})\pi(\bm{\gamma})\pi(\bm{\beta}) = \pi(\mu_0, \kappa| \bm{\gamma}, \bm{\beta}) \prod_{j=1}^q \pi(\gamma_j) \prod_{k = 1}^p \pi(\beta_k)$ where $\pi(\gamma_j) = (2\pi)^{-1}$, $\beta_k \sim N(0, \sigma^2$ with $\sigma^2$ chosen to be moderate (they default to $\sigma^2  = 1$ to the geometry of the posterior more favorable), and $(\mu_0, \kappa) | \bm{\gamma},\bm{\beta}$ gets the standard conjugate prior for the von mises distribution; \[\pi() \propto I_0(\kappa)^{-c}\exp\left[R_0\kappa \cos(\mu_0 - \eta_0)\right]\] which is non-informative for $c = 0, R_0 = 0$. 

Mulder and Klugkist also fully develop the range of inferences one would typically want to perform on a regression model including hypothesis testing for the parameter values (which includes model selection testing) via Bayes Factor and estimation via MCMC sampling \cite{mulder_bayesian_2017}. Since the dummy variables are specified for this model, ANOVA and ANACOVA are readily available for this model as well with the hypothesis testing again performed with Bayes factors.

\subsection{Mardia and Sutton Lineage}\label{mod_MS}
% \subsection*{Dependent Models for Observations which Include Angular Ones} - Kato, Shimizu (2008 - Journal of Statistical Planning and Inference) 
This first paper from Kato and Shimizu \cite{kato_dependent_2008}, bridges the two lineages but hews more towards the Mardia and Sutton side. Their paper has two parts: i) the introduction of a four-dimensional distribution with specified bivariate marginals on manifolds like two torii, cylinders, or discs and ii) two distributions on the cylinder based on the Johnson and Wehrly (1978) and Mardia and Sutton (1978) distributions. The densities proposed in (i) also are derived from the extended theory of copulas in \cite{carnicero_non-parametric_2013}\cite{fernandez-duran_models_2007}\cite{garcia-portugues_exploring_2013}. Both cylindrical models have generalized von Mises marginals for the circular components, one as an extension of Mardia and Sutton \cite{mardia_model_1978} and one as an extension of Johnson and Wehrly \cite{johnson_angular-linear_1978}. The generalized von Mises allows for skewness and bi-modality. The Johnson and Wehrly model \cite{johnson_angular-linear_1978} was proposed symmetric and unimodal circular marginals. 

Kato and Shimizu make a brief mention of incomplete data \cite{kato_dependent_2008}. To model pairs of wind directions observed at two different times in two different locations, we would use a torus-torus model, but if one observation is missing from the first location, we would then have a circle-torus model which could also be modeled by the generalized density they developed. However, this doesn't actually model the missing data as would be desired, but redefines the model to overlook the missingness a highly undesirable property.

Maximum likelihood is used for the estimation of the parameters, but only point estimates are discussed. The AIC is used to compare the model to the Mardia and Sutton model on the same skewed cylindrical data set \cite{kato_dependent_2008}. An LRT for the adequacy of fit and another LRT for independence were developed as well.

% \subsection*{The Joint Projected Normal and Skew-Normal: A distribution for Poly-cylindrical  Data}  -Gianluca Mastrantonio (2017 - Journal of Multivariate Analysis) 

Mastrantonio proposed the first poly-cylindrical distribution in the literature \cite{mastrantonio_joint_2018} meaning we have $p$ circular variates and $q$ linear variates. This new distribution is inherently Bayesian and flexible, it has interpretable parameters, and has an efficient estimation algorithm (MCMC) \cite{mastrantonio_joint_2018}. The circular marginal is the Projected Normal which can be bi-modal and skewed. The projected normal can be extended beyond a single circular random variable by taking $\mathbf{W} = (\mathbf{W}_1,...,\mathbf{W}_p) \sim N_{2p}(\bm{\mu}_w, \Sigma_w)$ where $\mathbf{W}_i = (W_{i1}, W_{i2})$ so that $\theta_i = \mathbf{W}_i / |\mathbf{W}_i| \sim PN(\bm{\mu}_{wi}, \Sigma_{i})$. The linear marginal is the skew-normal which is more flexible than the standard normal; $\mathbf{Z} \sim SSN_q(\bm{\mu}_z, \Sigma_z, \bm{\lambda})$ where $\bm{\mu}_z$, $\Sigma_z$ are defined as the mean vector and covariance matrix and $\bm{\lambda}$ is the skew parameter vector. Both have a multivariate augmented density representation based on the MVN. The joint cylindrical density is enabled by making the augmented normal representations dependent. \[f(\bm{\theta}, \mathbf{r}, \mathbf{z}, \mathbf{d}) = 2^q \phi_{p+q}\left(\begin{pmatrix}\mathbf{W}\\ \mathbf{z}\end{pmatrix} ~| \begin{pmatrix}\bm{\mu}_w + \mathbf{0}_{2p}\\ \bm{\mu}_z + \mathrm{diag}(\bm{\lambda})\mathbf{d}\end{pmatrix}, \begin{pmatrix} \Sigma_w & \Sigma_{wz} \\ \Sigma_{wz}' & \Sigma_z \end{pmatrix} \right) \phi_q(\mathbf{d} | \mathbf{0}_q, \mathbf{I}_q) \prod_{i=1}^p r_i \] This distribution is then called the Joint Projected Normal Skew Normal or JPNSN distribution \cite{mastrantonio_joint_2018}. Any subset of the random variables will also have the same cylindrical distribution. There is a parameter identification issue, but this is fixed with post processing. When working with the PN distribution, we can consider the joint distribution of $f(\bm{\theta}, \mathbf{R})$ where we have observations of $\bm{\theta}$ and treat $\mathbf{R}$ as latent observations.  

Identifiability issue comes from the fact that the MVN r.v. that defines the circular r.v. is equivalent to any multiple of that r.v. since the constants cancel when taking the inverse-tangent of the $(x,y)$ for each r.v.\cite{mastrantonio_joint_2018}. This issue extends to the JPNSN distribution since the marginal is PN. To remedy this, the variance of $W_{i2}$ is set to 1. This then complicates the MCMC sampling but is readily remedied.  

Inference is performed for JPNSN by using MCMC to sample from the posterior \cite{mastrantonio_joint_2018}. This gives provides us with readily available point and uncertainty estimates to create credible intervals or perform hypothesis testing with.

\subsection{Abe-Ley Model}\label{mod_AL}
% \subsection*{A Tractable Parsimonious and Flexible Model for Cylindrical Data, with Applications} – Abe, Ley (2017 – Economics and Statistics) 

Abe and Ley made a significant contribution to the literature with their proposal of the WeiSSVM (Weibull - Sine-skewed von Mises) distribution \cite{abe_tractable_2017}. This model is more flexible since it allows the circular marginal of $\theta$ to be skewed (although $Z$ must be non-negative as in Johnson and Wehrly model \cite{johnson_angular-linear_1978}). Non-negativity typically isn't an issue since we are often interested in the joint distribution of a direction and magnitude like speed or intensity. The new model is created by a power transformation of $Z \rightarrow Z^{1/\alpha}$ and perturbing the circular component. \[f(\theta, z) = \frac{\alpha \beta^{\alpha}}{2\pi \cosh \kappa} (1 + \lambda \sin(\theta - \mu)) z^{\alpha - 1} \exp\left[-(\beta z)^{\alpha} (1 - \tanh(\kappa)\cos(\theta - \mu))\right]\] Since the circular component allows for skewness it achieves a lot of the flexibility of the other modern Johnson and Wehrly lineage, copula-based models \cite{fernandez-duran_models_2007}\cite{garcia-portugues_exploring_2013}\cite{carnicero_non-parametric_2013} in a much simpler to implement form \cite{abe_tractable_2017}. Abe and Ley thoroughly developed the model for MLEs, LRTs, regression, sampling, and the Bayesian framework. Of the papers in this review, the Abe and Ley paper develops the inferential methods for their model most thoroughly. 

Parameter estimation can be performed via maximum likelihood or Bayesian inference from the posterior as developed by Abe and Ley \cite{abe_tractable_2017} or by Sadeghianpourhamami \cite{sadeghianpourhamami_bayesian_2019}. LRTs for independence and for the Johnson and Wehrly submodel are also proposed by Abe and Ley. Circular-linear and linear-circular regressions are also discussed by Abe and Ley. The regression equation for $Z$ is $E(Z | \theta) = \left[\beta (1 - \tanh(\kappa)\cos(\theta - \mu))^{1/\alpha}\right]^{-1} \Gamma(\alpha^{-1} + 1)$. While $\theta | Z \sim SSVM(\arg ((\beta z)^\alpha \tanh(\kappa) + i\lambda), A^{-1}(\rho))$ where $A^{-1}(x)$ is the inverse of $A(x) = I_1(x) / I_0(x)$ a ratio of the first order and zero-th order modified Bessel fucntions of the first kind and $\rho = \frac{I_1((\beta z)^\alpha \tanh(\kappa))}{(\beta z)^\alpha \tanh(\kappa) I_0((\beta z)^\alpha \tanh(\kappa))} \sqrt{(\beta z)^{2\alpha} \tanh^2(\kappa) + \lambda^2}$.

% \subsection*{Bayesian Cylindrical Data Modeling Using Abe-Ley Mixtures} – Sadeghianpourhamami, et  al (2018 – Applied Mathematical Modeling) 

A computationally focused paper proposing a Metropolis-Hastings algorithm for MCMC sampling to fit a mixture of Abe-Ley distributions was published the following year by Sadeghianpourhamami \cite{sadeghianpourhamami_bayesian_2019}. This is a proposed improvement over the EM algorithm proposed by Abe-Ley \cite{abe_tractable_2017}. This new method guarantees convergence to the global maxima and provides uncertainty estimates that the EM fails to provide. The MCMC approximates the posterior distribution so it reflects the multimodality of the posterior and avoids getting stuck in a local maxima since it doesn't do optimization. Uncertainty is captured by the posterior as well. 

\subsection{Other Models}\label{mod_other}

In this brief section, I'll quickly discuss two other cylindrical models that have been proposed over the last decade. The first model is from Nu\~{n}ez-Antonio et al \cite{nunez-antonio_bayesian_2011} and consists of a new formulation for the circular-linear regression problem where the Projected Normal is used for the circular marginal and the covariates $\mathbf{Z} \in \mathbb{R}^p$. The second model from SenGupta et al \cite{noauthor_model-based_nodate} considers the problem of clustering of cylindrical data points. 

The PN-based circular-linear regression model proposed by Nu\~{n}ez-Antonio et al \cite{nunez-antonio_bayesian_2011} is the first of its kind since the PN has a complicated density and proposed in a fully Bayesian way. Let $(\mathbf{Z}, \bm{\theta})$ be the cylindrical observations, then let $\theta_i | \mathbf{z}_i \sim PN(\bm{\mu}_i, \mathbf{I}_2)$ where $\mathbf{Y}_i = (Y_{i1}, Y_{2i})' \equiv \theta_i | \mathbf{z}_i \sim N_2(\mathbf{B}' \mathbf{z}_i, \mathbf{I}_2)$ where $\mathbf{z}_i \in \mathbb{R}^p$, $\mathbf{B} = (\bm{\beta}_1, \bm{\beta}_2) \in \mathbb{R}^{p \times 2}$, and $\bm{\beta_j} \in \mathbb{R}^p$ for $j = 1,2$ \cite{nunez-antonio_bayesian_2011}. Thus we can convert the projected normal regression into a standard multivariate regression problem. To fully specify the model in a Bayesian way, a normal prior is placed on $\mathbf{B}$. 

Nu\~{n}ez-Antonio et al \cite{nunez-antonio_bayesian_2011} also make the largest contribution to the missing data problem in the cylindrical model literature. Given missingness in the response $\theta$, the Gibbs sampler allows for data augmentation to treat the missing data as another unknown parameter vector that we can sample. This allows us to sample directly from the desired posterior given the complete data $\bm{\theta}_{comp} = (\bm{\theta}_{obs}, \bm{\theta}_{miss})$. While this is still an ad hoc method for treating the missing data problem, it is a more complete and thorough solution than the previous mentions in the literature which either side-step the issue or propose using a different model entirely \cite{kato_dependent_2008}. 

Inference on Nu\~{n}ez-Antonio et al's PN regression model is then performed using standard Bayesian analysis of the posterior approximated by Gibbs sampling \cite{nunez-antonio_bayesian_2011}. Sampling is performed by using latent variables to treat the problem as a multivariate regression. Credible intervals, model selection, and point estimates of the parameters in $\mathbf{B}$ are discussed \cite{nunez-antonio_bayesian_2011}. 

% \subsection*{Model-Based Clustering for Cylindrical Data} - SenGupta, Roy, Chattophadhyay (2020 - Book: Advances in Statistics - Theory and Applications) 

In this paper (published as section of a book chapter) \cite{noauthor_model-based_nodate}, SenGupta et al are concerned with the clustering of cylindrical points and specify conditional distributions for $\theta$ on $x$ and $x$ on $\theta$. Two models are proposed (a) marginal dist of $x$ and conditional of $\theta$ on $x$ (b) marginal dist of $\theta$ and conditional of $x$ on $\theta$. Both fit with the EM algorithm for curved EF models. Provides a list of previously discussed cylindrical models in the introduction.  

The marginal theta and conditional $x$ on $\theta$ model uses a von Mises for the $\theta$ marginal and Normal with the regression for the mean (basically identical to Mardia and Sutton's model) \cite{mardia_model_1978}. The mixture combines $p$ component distributions and is fit with EM. The application section again uses the Kew data set from Mardia \cite{mardia_model_1978}. 
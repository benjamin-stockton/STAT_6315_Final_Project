\section{Conclusions}\label{conc}

The cylindrical statistics literature has grown much deeper in the past two decades as more attention is paid to the issues inherent to directional data and as computational power has progressed to allow for higher complexity in the models. The two lineages initiated in 1978 by Johnson and Wehrly \cite{johnson_angular-linear_1978} and Mardia and Sutton \cite{mardia_model_1978} have grown greatly, particularly the Johnson and Wehrly model's more convenient extension to circular-linear regression. There have also been models proposed that have bridge the gaps between the two lineages \cite{kato_dependent_2008}, \cite{mastrantonio_joint_2018}. The introduction of the Abe-Ley distribution in 2017 \cite{abe_tractable_2017} has allowed for another burst of attention to cylindrical modeling and new applications of the techniques.

In my research over the next few years, I want to develop missing data methods for directional data starting with the cylindrical data case. As is clear from the very sparing and short references to missing data analysis the problem is both unaddressed and conceptually challenging. While there's a robust literature for the analysis of missing data in the linear setting, the idiosyncrasies of circular and spherical random variables make the analysis of missing data particularly challenging. Most of the methods developed so far have been ad hoc as they were for linear data prior to the foundational papers from Rubin \cite{rubin_inference_1976} and Dempster et al \cite{dempster_maximum_1977}. 
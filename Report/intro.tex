\section{Introduction}\label{intro}

% - What are Directional Stats?
Directional statistics consists of the analysis of random variables distributed on the unit circle or unit sphere of higher dimension. Analysis on these nonlinear manifolds require explicit and careful consideration. Some examples of directional data include directly measuring angles such as the wind, measuring the angle of an axis like the tilt of the Earth's axis, or the direction of a point in $\mathbb{R}^p$ rather than its vector value as in text analysis or genomic data when the vectors are scaled to be unit length \cite{mardia_directional_1999}, \cite{ley_applied_2018}. 

The classic statistical techniques developed for linear data generally do not apply to data collected from a directional context as the linear techniques do not respect the geometric properties of the circle or sphere. As an example, consider two points on $S^1$, $\theta_1 = \frac{\pi}{6}$ and $\theta_2 = \frac{11\pi}{6}$. The arithmetic mean, upon which many linear methods are based, is $\frac{1}{2}(\frac{\pi}{6} + \frac{11\pi}{6}) = \pi$ which lies on the opposite side of the unit circle from the two points. Intuitively we would expect the mean of two angles to lie on the shortest arc between them, not the longest \cite{mardia_directional_1999}. Directional methods handle these issues and produce results respecting the geometric surfaces from which the data are drawn.

% Brief review of missing data methods and why they're difficult to use/incompatible with directional data
In the broader statistical literature beyond directional statistics, there has also been tremendous growth and development of missing data methods and analysis over the past 50 years. These methods largely began with Dempster et al's work formalizing the EM algorithm \cite{dempster_maximum_1977} and Rubin's development of Multiple Imputation methods \cite{rubin_inference_1976}. While the framework for the EM algorithm is general enough to accommodate directional random variables to some extent, Rubin's rules for estimate combination are grounded in linear random variables. Specifically, Rubin's rules require that the estimates have a defined variance, something that is undefined for circular random variables, such as the sample mean direction. This complication results in the literature have sparingly few treatments of missing or incomplete data, and when the topic does arise, ad hoc methods are used to handle the incompleteness \cite{nunez-antonio_bayesian_2011} \cite{anderson-sprecher_state-space_2021} \cite{kato_dependent_2008}. A few instances of the ad hoc methods will be discussed later in this review.

% Distributions on the circle

Over the past 120 years many distributions on the circle have been proposed. Most central to directional data analysis is the normal distribution's analogue the von Mises distribution \cite{mardia_directional_1999}. A circular random variable $\theta \sim M(\mu, \kappa)$ has a mean direction parameter $\mu$, and a concentration parameter $\kappa$. The density of $\theta \in (0, 2\pi]$ is \[f(\theta) = \{I_0(\kappa)\}^{-1} \exp\left[\kappa \cos(\theta -\mu)\right]\] where $I_0(x)$ is the 0-th Bessel function of the first kind. This distribution is symmetric and unimodal on the circle and for large $\kappa$ can be approximated by the normal distribution \cite{mardia_directional_1999}. The von Mises has also been generalized to allow for skewness \cite{abe_tractable_2017}. Beyond the von Mises, there is also the Projected Normal distribution. Let $\mathbf{X} \sim N_2(\bm{\mu}, \bm{\Sigma})$, then $\mathbf{X} / |\mathbf{X}| \sim PN(\bm{\mu}, \bm{\Sigma})$, so that the projected normal random variable is expressed in Cartesian coordinates as a unit vector pointing in the direction of the unobserved random vector $\mathbf{X}$. To convert between angular values and Cartesian coordinates, we can use this identity \[\theta \equiv (\cos\theta, \sin\theta) = (x,y)\] The Project Normal distribution is more flexible than the von Mises as it allows for bimodality \cite{mardia_directional_1999}. There are many other distributions on the circle, but they will be discussed as necessary later in this review.

% Cylindrical models

Cylindrical data combine a directional component, most often a circular random variable, and a linear component, either on $\mathbb{R}$ or a concave subset of $\mathbb{R}$. Observations are then a pair $(\theta_i, z_i)$ of cylindrical coordinates where $\theta_i$ is an angle and $z_i$ is a "height" or magnitude associated with the angle $\theta_i$. Three relationships may then be of interest: (i) the joint distribution on the cylinder of $(\bm{\theta}, \mathbf{z})$, (ii) the conditional distribution of $\bm{\theta}$ given $\mathbf{z}$ (in the regression context, this is referred to as the circular-linear regression model), or (iii) the conditional distribution of $\mathbf{z}$ given $\bm{\theta}$. Relationship (i) can be useful for modeling a complex relationship such as wind or wave currents \cite{abe_tractable_2017}. As such the cylindrical models become very useful in spatial contexts where we consider a lattice of cylindrical random variables rather than a single one \cite{ranalli_segmentation_2018}. The conditional relationships of (ii) and (iii) are analogous to regression and as such are modeled in equivalent ways. In particular, relationship (iii) can be modeled using GLMs where we use $\cos k \bm{\theta}$ and $\sin k\bm{\theta}$ as covariates for some values of $k = 1,2,...$ \cite{mardia_model_1978} \cite{anderson-cook_second_2000}. 


% - Wind/Cylindrical models
For a more specific look at an example of cylindrical data, we can consider modeling the wind. In its simplest form, this means measuring the wind direction and wind speed jointly. The wind direction is measured as an angle so it is directly from $S^1$ while the speed is non-negative and from $R^+$. The joint distribution of these values lies on the cylinder $S^1 \times R^+$. Many models have been developed to consider this kind of data including Mardia and Sutton \cite{mardia_model_1978}, Fisher and Lee \cite{fisher_regression_1992}, Abe and Ley \cite{abe_tractable_2017}, etc. Models for wind data explicitly have also been developed by others. While not all of these were originally proposed as Bayesian models, many have been adapted to Bayesian analysis and inference while new models have also been developed specifically for Bayesian analysis and inference.

This literature review is organized as follows. In section \ref{intro}, we have discussed preliminary background information on directional statistics, cylindrical models, and several applications of each. In section \ref{hist_sec}, we explore the historical development of cylindrical models which were mainly created in two frequentist frameworks. In section \ref{mod_sec}, we consider the cylindrical models developed over the past 20 years which have branched out into Bayesian and non-parametric frameworks. In section \ref{app_sec}, we discuss several papers providing example analyses of cylindrical data including wind, wave currents, and animal movement telemetry data. In section \ref{ex_sec}, a brief example analysis of wind and pollution data from Milwaukee, WI is shared. The report concludes with a brief discussion and conclusion in section \ref{conc}. 

Throughout the paper, I have used the standardized notation of $Z$ or $\mathbf{Z}$ for linear random variables or vectors/matrices and $\theta$ for circular random variables along with their respective pdfs and cdfs $f_Z(z), F_Z(z), f_\mathbf{Z}(\mathbf{z}), F_\mathbf{Z}(\mathbf{z}), f_\theta(\theta)$, and $F_\theta(\theta)$. As usual $\phi(.)$ and $\Phi(.)$ refer to the PDF and CDF of the standard normal. 